\chapter{Ejercicio Adicional: \textbf{Exploración del Modelo de Mezcla de Gaussianas (GMM)}}
\label{chapter:tarea_d}

Esta sección es completamente voluntaria y puede realizarla para mejorar su puntuación en la práctica.

\subsection*{Objetivo}
Investigue cómo funciona el modelo de mezcla de gaussianas (GMM) para mejorar la detección en condiciones de iluminación cambiantes.

\begin{enumerate}
    \item \textbf{Implementación del GMM}: Utilice \texttt{cv2.createBackgroundSubtractorMOG()} y ajuste el parámetro \texttt{history} para observar cómo cambia la detección en función de la duración de la memoria del fondo.
    \item \textbf{Comparación con MOG2}: Observe las diferencias en la sensibilidad a las sombras y los cambios de iluminación. Pruebe con vídeos que incluyan cambios graduales de iluminación y objetos que se detienen temporalmente.
\end{enumerate}

\section*{Preguntas}
\addcontentsline{toc}{section}{Preguntas}

\vspace{5mm}
\begin{tcolorbox}[colback=gray!10, colframe=gray!30, coltitle=black, title=Pregunta D.1, halign=left]
¿Qué ventajas observa en \texttt{createBackgroundSubtractorMOG} en comparación con \texttt{createBackgroundSubtractorMOG2}?
\end{tcolorbox}

\vspace{5mm}
\begin{tcolorbox}[colback=gray!10, colframe=gray!30, coltitle=black, title=Pregunta D.2, halign=left]
¿Cómo afecta el parámetro \texttt{history} al rendimiento de detección en escenas con objetos que aparecen y desaparecen?

\vspace{3mm}
Al finalizar este ejercicio, agregue las observaciones y las imágenes generadas en el informe final de la práctica para una evaluación completa de los resultados.
\end{tcolorbox}